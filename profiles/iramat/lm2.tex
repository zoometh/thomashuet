
%!TEX root = ./lm2.tex
\documentclass[12pt]{article}
\usepackage[margin=1in]{geometry} % to change the page dimensions
\usepackage{graphicx} % for including graphics
\usepackage{amsmath} % for better math formatting
\usepackage{amssymb}
\usepackage{enumitem} % to customize list
\usepackage{fancyhdr} % for better header layout

% Set up the header and footer
\pagestyle{fancy}
\fancyhf{} % clear all header and footer fields
\fancyfoot[R]{\thepage} % page number on the right of the footer
\renewcommand{\headrulewidth}{0pt} % remove the header rule
\renewcommand{\footrulewidth}{0pt} % remove the footer rule

\setlength\parindent{0pt} % Removes all indentation from paragraphs

\begin{document}

Mesdames, Messieurs les membres du jury,

L'IRAMAT ouvre un poste d'Ingénieur-e en production, traitement et analyse de données dans un contexte plus général
d'ouverture, d'internationalisation et de massification des données archéologiques. Je suis depuis 3 ans responsable d'une
base de données (BD) pour la documentation et gestion du patrimoine archéologique en danger, à l'Université d'Oxford.
Auparavant j'avais travaillé en France et en Espagne très généralement dans les aspects de gestion, d'analyse et de
modélisation des données. Je souhaite démontrer ma capacité à répondre aux compétences demandées par ce
poste:

\section*{Activités}

\textbf{Prendre en charge la mise en place, la structuration et la maintenance des bases de données intégrant des données de
nature variée et hétérogène (quantitatives, catégorielles, spatiales...) produites par les membres de l'unité et en lien
avec les solutions idoines offertes par l'Infrastructure de recherche IR Huma-Num ;}
Mon travail en tant que chercheur et gestionnaire de la BD sémantique géoweb open-access EAMENA couvre des aspects
d'administration système, d'administration et de gestion BD, et des aspects d'analyses statistiques et de publication. Les
données que je suis chargé d'intégrer à EAMENA peuvent être structurées mais présentant des caractéristiques
idiosyncratiques (ex: archives XLSX des chercheurs) que des données déjà LOUD (Linked Open Useful Data). Je
maîtrise le cycle de déploiement des BDs: du modèle conceptuel au déploiement web en passant par les permissions
utilisateurs, sauvegardes, etc. Je connais l'écosystème de la recherche au Royaume-Uni, en France, mais aussi
européenne. J'ai par exemple utilisé différents services de l'IR* Huma-Num, ses outils collaboratifs. Je connais,
directement ou indirectement, plusieurs de ses consortiums.
\smallbreak
\textbf{Concevoir et mettre en œuvre les outils nécessaires à la diffusion et à l'archivage des données archéométriques
produites au laboratoire selon les principes des Linked Open Data ;}
J'ai une bonne expérience de l'Open Data en ayant publié ces trois dernières années trois data papers. Je suis aussi l'un
des deux référents Chercheurs au niveau universitaire (Université d'Oxford) pour l'Open data. J'ai aligné les périodes
chrono-culturelles d'EAMENA sur le gazetteer PeriodO, j'ai développé un plugin qui permet d'exporter le résultat d'une
recherche depuis la BD directement dans Zenodo (communauté EAMENA) en traitant les métadonnées de l'export à la
volée et en assurant la mise en relation des jeux de données (LOD) dans Zenodo. J'ai travaillé sur plan de gestion des
données (PGD) de l'ANR Itineris et identifié les alignements nécessaires (isostandards, ontologies, vocabulaires) pour
permettre la FAIRisation des données du projet. J'utilise GitHub et GitLab.
\smallbreak
\textbf{Assurer l'intégration des travaux du laboratoire aux projets collaboratifs (nationaux et internationaux) dans lesquels
s'insèrent les recherches de l'unité (par ex. Nomisma.org pour la numismatique, TerraLID pour les rapports isotopiques,...) ;}
Je travaille au quotidien au mappage des données laboratoires vers des standards et isostandards. Je suis associé à un
projet (CENTAURO) et une ANR (Itineris) qui traitent des données isotopiques et je connais les plateformes de partage de ces
données (ex: IsoArcH, TerraLID), de même que pour la structuration et la FAIRisation des données numismatiques j'ai
connaissance de l'état de la recherche internationale (Nomisma) et nationale (projet ATMOCE et BD AeMa).
Je travaille à l'interopérabilité de projets universitaires utilisant Arches (le logiciel d'EAMENA), notamment sur le partage de
nos données de références: graphes (tables et relations entre les tables) et vocabulaires, ainsi qu'une partie de notre
développement logiciel.
J'ai développé un connecteur, dans le cadre du package R C14bazAAR, pour collecter les données radiocarbones du
dataset NeoNet.
\smallbreak
\textbf{Accompagner le développement de nouvelles méthodologies et de traitements des données hétérogènes (chimie,
archéologie, histoire...) en lien avec les approches statistiques et numériques de l'unité ;}
J'ai travaillé au carrefour de problématiques archéologiques et patrimoniales très différentes. Mes connaissances et
compétences couvrent déjà une large partie de l'hétérogénéité des données chronologiques, spatiales ou
thématiques. J'ai déjà de l'expérience avec des natures de données directement en lien avec les problématiques de
l'IRAMAT, notamment: l'analyse d'images et de surfaces, du micro (microscopie) au géographique (SIG), la 3D et web3D
(suivant les mêmes échelles), la création logicielle (Python et R), la création d'interfaces interactives pour la "fouille de
données" (data mining, par exemple: analyses factorielles, ou webmapping) avec une application R Shiny hébergée en
ligne ou localement
\smallbreak
\textbf{Participer aux principaux réseaux nationaux et internationaux en lien avec la gestion et l'analyse de données
archéométriques, historiques et archéologiques et constituer un contact privilégié pour le laboratoire ;}
J'ai été ou suis impliqué dans différents projets: édition 2023 du GMPCA (session "gestion des jeux de données"),
celle avenir du GMPCA 2025 (même thème), chapitre français du Computer Application for Archaeology (CAA), Special
Interest Group on Scientific Scripting Languages in Archaeology (SIG-SSLA), entre autres. Mon expérience professionnelle en
Espagne puis au Royaume-Uni fait aussi que j'ai un réseau étendu de collaborations.
\smallbreak
\textbf{Assurer une veille scientifique et méthodologique ;}
Je le fais régulièrement, notamment en participant à de nombreuses formations, à des colloques internationaux, à
différents comités de relecture, entre autres.
\smallbreak
\textbf{Former et encadrer des étudiants et des stagiaires ;}
J'ai été impliqué dans un comité de suivi d'une thèse d'archéologie et tuteur d'un M2. J'ai donné des enseignements
universitaires, je participe aux formations des utilisateurs de la BD EAMENA et à celles des gestionnaires des clones nationaux
d'EAMENA.
\smallbreak
\textbf{Participer ou coordonner des réponses à des appels à projets régionaux, nationaux ou internationaux ;}
En 2024, j'ai participé au renouvellement du financement triennal du projet EAMENA, à l'écriture d'un projet ANR, ou
d'autres projets collectifs de recherche notamment en Espagne. J'ai personnellement obtenu des financements pour
l'organisation de workshops, de formation ou de défraiement.
\smallbreak
\textbf{Valoriser les réalisations et résultats dans des colloques nationaux et internationaux, des publications écrites, des
rapports etc. et participer aux actions de diffusion vers des publics variés.}
Je participe régulièrement aux colloques internationaux et nationaux, j'ai développé des solutions web interactives
comme le webmapping et les documents interactifs de données. J'ai participé à de nombreux rapports, publié des articles
et des rapports sur des sujets très différents les uns des autres, dans des revues nationales et internationales. J'ai
communiqué à l'adresse d'un public général.

\section*{Compétences :}

\subsection*{Savoirs :}

\textbf{Maîtrise des systèmes de gestion de base de données en particulier PostgreSQL et MySQL ;}
Je travaille depuis plus de 10 ans avec Postgres/PostGIS, et depuis 3 ans dans un contexte geoweb
(Linux/Apache/Postgres/Django/Python).
\smallbreak
\textbf{Savoir développer des interfaces d'interrogation et de diffusion des BDD (HTML, PHP, CSS, javascript). Une connaissance
des langages du webGIS (openLayers, leaflet) et/ou de RShiny serait un plus ;}
J'ai développé plusieurs packages en R qui utilisent R Shiny et des bibliothèques JavaScript (notamment Leaflet et Plotly).
L'une de ces applications (NeoNet) est en ligne, sur un serveur universitaire et j'en assure régulièrement la mise à jour
(connexion SSH, RStudio Server, etc.). Je sais que l'IR* Huma-Num maintient des serveurs R, RStudio et R Shiny, ce qui serait
très intéressant à coupler avec leur services GitLab pour une exposition interactive et dynamique des données et analyses
effectuées au laboratoire. Je participe au maintien du site web du projet EAMENA, je connais donc les principales syntaxe du
web, les protocoles réseaux, l'utilisation des API, entre autres.
\smallbreak
\textbf{Très bonne connaissance de la réglementation en matière de stockage, sauvegarde, diffusion et protection des données,
des normes de documentation de données et de métadonnées (protocole OAI-PMH), en assurant le respect des principes
FAIR ;}
Je connais les politiques de responsabilité des données, les standards et outils de stockage à court, moyen et long terme,
l'usage des licences. Comme mentionné ci-dessus, à propos de Zenodo, je sais moissonner les metadata des entrepôts de
données OAI-PMH et me servir des API.
\smallbreak
\textbf{Maîtrise de logiciels SIG prioritairement QGIS appréciée ;}
Je maîtrise QGIS que je couple a Postgres et à des scripts Python. 
\smallbreak
\textbf{Connaissance d'au moins un des deux langages Python et R ;}
Je maîtrise les deux langages. Je privilégie le langage R pour les analyses statistiques, les flux de travaux qui impliquent
des structures de données chercheur (ex: XLSX) ou les chercheurs eux-mêmes (R Shiny). Je privilégie Python pour
l'infrastructuration des données (c'est le langage de conception de la BD EAMENA). J'utilise R Shiny et les Jupyter Notebooks
(Ipython) pour l'interactivité.
\smallbreak
\textbf{Connaissance des analyses statistiques et/ou numériques pour le traitement des données appréciée ;}
Je maîtrise les statistiques descriptives, exploratoires (analyses multifactorielles) et confirmatoires (tests). J'ai publié des
études statistiques et spatiales portant sur des sujets très différents dans des revues internationales et nationales.
\smallbreak
\textbf{Connaissance générale des méthodes d'analyses physico-chimiques appréciée ;}
J'ai commencé a travaillé sur les isotopes du plomb dans le cadre de l'ANR Itineris. Je participe à l'analyse des isotopes de
l'azote et l'oxygène dans le cadre du projet CENTAURO. J'ai des notions en sciences des sols, sciences des matériaux,
analyses de surface, entre autres.
\smallbreak
\textbf{Langue anglaise : B2 (cadre européen commun de référence pour les langues).}
J'ai un bon niveau en anglais et espagnol (B2/C1).

\subsection*{Savoir-faire :}

\textbf{Maîtrise de la conception, du déploiement et de la maintenance des bases de données spécifiques à
l'archéométrie (définition de modèles conceptuels de données, développement d'interfaces de dépôt et de
consultation des données) ;}
J'ai participé à l'élaboration du module "Modélisation et analyses des données" de l'ANR Itineris, notamment autour du
PGD et de la modélisation conceptuelle de la BD.
\smallbreak
\textbf{Concevoir des applications en ligne (de type WEBSIG) qui implémentent les inférences statistiques développées au
sein de l'unité ;}
Je gère une BD géoweb (EAMENA) et une application websig R Shiny/Leaflet (NeoNet) qui sont en ligne. Je maîtrise les
outils du géoweb (GeoServer, service de cartes WMS, WFS, entre autres).
\smallbreak
\textbf{Savoir élaborer un cahier des charges, formaliser des techniques et des méthodes, et rédiger des rapports et des
documents.}
Dans le cadre d'EAMENA, je suis amené à définir et superviser les développements informatiques des consultants du
projet. Je contribue aux rapports d'activité semestriels et annuels du projet.
\smallbreak
\subsection*{Savoir être :}
\textbf{Qualités relationnelles ;}
Je suis autonome dans mon travail mais j'apprécie beaucoup le travail en groupe.
\smallbreak
\textbf{Capacité à travailler en équipe ;}
Je pense qu'à plusieurs on va plus vite et plus loin. C'est pour cela que je postule à ce poste d'IR.
\smallbreak
\textbf{Capacité à travailler en partenariat.}
Je le fais régulièrement. Je suis impliqué dans différents projets, groupes de recherches, comité de relecture, etc.
\bigbreak
Jusqu'ici, mes compétences dans la gestion et modélisation des données m'ont permis de travailler au carrefour de
nombreuses problématiques archéologiques. Aujourd'hui, je souhaite reprendre des recherches archéologiques en
France, particulièrement dans un contexte interdisciplinaire où les archéomatériaux et le traitement de l'information
archéologique pourraient jouer un rôle clef.

\end{document}