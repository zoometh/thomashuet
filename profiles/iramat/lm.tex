
\documentclass[12pt]{article}
\usepackage[margin=1in]{geometry} % to change the page dimensions
\usepackage{graphicx} % for including graphics
\usepackage{amsmath} % for better math formatting
\usepackage{amssymb}
\usepackage{enumitem} % to customize list
\usepackage{fancyhdr} % for better header layout

% Set up the header and footer
\pagestyle{fancy}
\fancyhf{} % clear all header and footer fields
\fancyfoot[R]{\thepage} % page number on the right of the footer
\renewcommand{\headrulewidth}{0pt} % remove the header rule
\renewcommand{\footrulewidth}{0pt} % remove the footer rule

\setlength\parindent{0pt} % Removes all indentation from paragraphs

\begin{document}

\begin{center}
    \Large\textbf{Thomas Huet} % Your name
\end{center}

\begin{flushright}
  \today \\[12pt] % Date
  8 Chestnut Rd, Botley \\
  OX2 9EA, Oxford \\
  United Kingdom\\
  Email: thomashuet@gmail.com \\
  Phone: +44(0)7 518 152 642
\end{flushright}

\textbf{Recipient Name} \\
Title \\
Company Name \\
Street Address \\
City, State, Zip Code \\

Mesdames, Messieurs les membres du jury,

\vspace{12pt} % Add some vertical space
% Body of the letter
Archéologue et travaillant quotidiennement avec des outils informatiques (programmation, bases de données, LOD, etc.). 

Je suis depuis 3 ans responsable de la base de données EAMENA, une base de données sémantique géoweb pour la documentation et gestion du patrimoine archéologique en danger, à l'Université d'Oxford. Auparavant j'avais travaillé plus proprement dans un contexte recherche archéologique aussi bien en France (CNRS, Universités) qu'en Espagne (Université) et ce a differents postes (Technicien, Ingénieur d'etude et Ingénieur de Recherche et de Chercheur). Dans ces differentes missions et tres generalement dans les aspects de gestion et d'analyses des donnees.

Mon expérience couvre donc différents contextes et problématiques académiques dans un contexte d'ouverture et d'internationalisation de la recherche académique. Je suis particulièrement intéressé par l'ouverture d'un poste d'Ingénieur-e en production, traitement et analyse de données au laboratoire IRAMAT car.

Reprenant la fiche de poste

Activités :
 
- Prendre en charge la mise en place, la structuration et la maintenance des bases de données intégrant des données de nature variée et hétérogène (quantitatives, catégorielles, spatiales...) produites par les membres de l'unité et en lien avec les solutions idoines offertes par l'Infrastructure de recherche IR Huma-Num ;
Mon travail pour EAMENA couvrent des aspect administrateurs systeme, administrateur et gestionnaire base de donnees, et des aspects d'analystes et de recherche. Je travaille directement avec des serveur web (SSH, Linux), je connais pour les avoir utilises les services Huma-Num (Nakala, GitLab, etc.). Les donnees dont j'ai la charge represenetent aussi bien des donnees structurees mais relativement idiosynchratiques (ex: archives XLSX des chercheurs) que des donnees LOUD 
- Concevoir et mettre en œuvre les outils nécessaires à la diffusion et à l'archivage des données archéométriques produites au laboratoire selon les principes des Linked Open Data ;
Dans le cadre de ma participation a l'ANR Itineris, portant entre autre sur l'analyse isiotopique, j'ai participe au plan de gestion des donnees et identifier en amont les alignements necessaires (isostandards, ontologies) pour permettre la FAIRisation des donnees du projet avec les standards internationaux. Par ailleurs je travaille au quotidient avec les problematiques FAIR et LOD. J'ai par exemple developpe un plugin qui permet d'ecrire directement un jeu de donnees dans Zenodo en traitant les metadonnees a la volee, j'ai aligne les periodes chrono-culturelles d'EAMENA sur le gazetteer PeriodO. 
- Assurer l'intégration des travaux du laboratoire aux projets collaboratifs (nationaux et internationaux) dans lesquels s'insèrent les recherches de l'unité (par ex. Nomisma.org pour la numismatique, TerraLID pour les rapports isotopiques,...) ;
J'ai paerticiper a l'internationalisation de la BD EAMENA, du projet ANR Itineris. Je sais me servir des API, gerer les permissions CRUD (create, read, update, delete). Plus generalement, pour avoir travailler dans des contextes tres differents, je sais mapper les donnees recherches vers differents depots de donnnees.
- Accompagner le développement de nouvelles méthodologies et de traitements des données hétérogènes (chimie, archéologie, histoire...) en lien avec les approches statistiques et numériques de l'unité ;
J'ai travaille au carrefour de problematiques archeologiques et patrimoniales tres differentes. Mes connaissances et competences couvrent deja une large partie de l'heterogeite des donnees comme le montre mes publications.
- Participer aux principaux réseaux nationaux et internationaux en lien avec la gestion et l'analyse de données archéométriques, historiques et archéologiques et constituer un contact privilégié pour le laboratoire ;
Je suis implique dans differents projets comme l'edition 2023 du GMPCA (session gestion des jeux de donnees), celle avenir du GMPCA 2025 (meme sujet), le chapitre francais du CAA (Computer Application for Archaeology), l'edition d'un numero special de la revue Archaeometry (Oxford) sur la modelisation chronologique. Je suis en relation avec le reseau CAI-RN. Si ma candidature est retenue, je prendrai une part active dans les reseaux nationaux et internationaux
- Assurer une veille scientifique et méthodologique ;
Je le fais regulierement, notamment en particpant a de nombreuses formations (voir mon CV), en participant a des colloques internationaux (notamment le CAA), en etant implique dans des comites de relecture aussi bien sur les aspects logiciels (JOSS) que donnnees (JOAD).
- Former et encadrer des étudiants et des stagiaires ;
Je le fais aussi de maniere reguliere: je supervise les developpement informatiques des consultatnts d'EAMENA, de stagiaires, la formation des utilisateurs de la BD EAMENA. J'ai ete par ailleurs implique dans le comite de suivi d'une these d'archeologie. J'ai organise plusieurs formation en interne pour les permanents de l'equipe (entre autres: langage R, utilisation de GitHub). J'ai donne des enseignements et animes ateliers a l'Universite d'Oxford, au CSIC de Barcelone, dans des UMR de Nice et Montepellier.
- Participer ou coordonner des réponses à des appels à projets régionaux, nationaux ou internationaux ;
J'ai participe au succes du renouvellement du financment du projet EAMENA, a l'ecriture de projet ANR, ou d'autres projets collectif de recherche qui ont ete ensuite finances. Ces deux dernieres annees j'ai personellemt obtenu des financements pour l'organisation de workshops. Si ma candidature est retenue, j'apporterai mes competences a l'ecriture de projets aux membre du laboratoire qui le necessiteront, notamment sur les aspects de developpement informatique, et je prendrai une attitude volontariste pour coordoner des propositions de projet en vue de leur financement. 
- Valoriser les réalisations et résultats dans des colloques nationaux et internationaux, des publications écrites, des rapports etc. et participer aux actions de diffusion vers des publics variés.
Je participe regulierement aux colloques internationaux et nationaux (organisation de sessions, comminications, relecture des actes, etc.), je developpe pour le web et suis donc capable de mettre en place des solutions interactives comme le webmapping (ou geoweb) et les documents de donnees (graphiques interactifs). J'ai publie des articles et des rapports sur des sujets tres differents les uns des autres, dans des revues nationnales et internationales. 


Compétences :
 
Savoirs :
- Maîtrise des systèmes de gestion de base de données en particulier PostgreSQL et MySQL ;
Je travaille depuis plus de 10 ans avec PostgreSQL/PostGIS, et depuis 3 ans dans un contexte geoweb. Je connais par ailleurs les architectures LAMP (Linux, Apache, MySQL, PHP/Perl/Python). Je maitrise l'ensemble des aspects "donnees" en manipulant des resources legeres (ex: XLSX) que lourdes (base de donnees).
- Savoir développer des interfaces d'interrogation et de diffusion des BDD (HTML, PHP, CSS, javascript). Une connaissance des langages du webGIS (openLayers, leaflet) et/ou de RShiny serait un plus ;
J'ai developpe plusieurs logiciels en R qui utilisent R Shiny et des bibliotheques JavaScript (notamment Leaflet). L'une de ces applications (NeoNet) est en ligne, sur une serveur institutionel (Universite de Pisa) et j'en assure regulierement la mise a jour. 
- Très bonne connaissance de la réglementation en matière de stockage, sauvegarde, diffusion et protection des données, des normes de documentation de données et de métadonnées (protocole OAI-PMH), en assurant le respect des principes FAIR ;
- Maîtrise de logiciels SIG prioritairement QGIS appréciée ;
- Connaissance d'au moins un des deux langages Python et R ;
Je maitrise les deux langages. Je privilegie R pour l'analyse statistique des donnees, les flux de travaux qui impliquent des structures de donnees chercheur (ex: XLSX). J'ai developpe plusieurs package en R. Je privilegie Python pour l'infrastructuration des donnees (c'est le langage de conception de la BD EAMENA), le traitement des images (OpenCV), et dans le futur pour l'apprentissage machine (PyTorch, Keras)
- Connaissance des analyses statistiques et/ou numériques pour le traitement des données appréciée ;
Je maitrise les statistiques descriptives, exploratoires (analyses multifactorielles) et confirmatoires (tests). J'ai publie des etudes statistiques portant sur des sujets tres differents. 
- Connaissance générale des méthodes d'analyses physico-chimiques appréciée ;
J'ai commence a travaillers sur les isotopes du cuivre dans le cadre de l'ANR Itineris. Je participe a l'analyses des isotope de l'azote et oxygène dans le cadre du projet CENTAURO. 
- Langue anglaise : B2 (cadre européen commun de référence pour les langues).
Je parle l'anglais et l'espagnol de maniere correcte.

Savoir-faire :
- Maîtrise de la conception, du déploiement et de la maintenance des bases de données spécifiques à l'archéométrie (définition de modèles conceptuels de données, développement d'interfaces de dépôt et de consultation des données) ;
Comme precedemment dit, j'ai participe a l'elaboration du projet ANR Itineris, notamment autour du PGD et de la conception de la base de donnees.
- Concevoir des applications en ligne (de type WEBSIG) qui implémentent les inférences statistiques développées au sein de l'unité ;
Je gere une base de donnees (EAMENA) et une application (NeoNet) qui sont en ligne
- Savoir élaborer un cahier des charges, formaliser des techniques et des méthodes, et rédiger des rapports et des documents.
J'ai deja ces experiences

Savoir être :
- Qualités relationnelles ;
Je suis quelqu'un de calme, j'apprecie aussi bien le travail en groupe que les periodes de travail personnel
- Capacité à travailler en équipe ;
J'ai travaille au carrefours de nombreuses problematiques, et j'apprecie beaucoup cela. Quand necessaire, je sais reformler les besoins des chercheurs en terme de besoins de structuration des donnnees. Je sais faire aussi le pont entre des archeologues, les domaines des archeosciences et les specilistes de l'informatique.
- Capacité à travailler en partenariat.
Je le fais regulierement. Par exemple le projet EAMENA regroupe 3 differentes universites anglaises et a des partenariat en Jordanie, Palestine, Lybie, Algerie, etc. 

\vspace{12pt} % Add some vertical space
Sincerely,

\vspace{48pt} % Space for your signature

John Doe % Your name again for closure

\end{document}
