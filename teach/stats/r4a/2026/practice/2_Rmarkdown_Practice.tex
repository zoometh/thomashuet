% Options for packages loaded elsewhere
\PassOptionsToPackage{unicode}{hyperref}
\PassOptionsToPackage{hyphens}{url}
\documentclass[
]{article}
\usepackage{xcolor}
\usepackage[margin=1in]{geometry}
\usepackage{amsmath,amssymb}
\setcounter{secnumdepth}{-\maxdimen} % remove section numbering
\usepackage{iftex}
\ifPDFTeX
  \usepackage[T1]{fontenc}
  \usepackage[utf8]{inputenc}
  \usepackage{textcomp} % provide euro and other symbols
\else % if luatex or xetex
  \usepackage{unicode-math} % this also loads fontspec
  \defaultfontfeatures{Scale=MatchLowercase}
  \defaultfontfeatures[\rmfamily]{Ligatures=TeX,Scale=1}
\fi
\usepackage{lmodern}
\ifPDFTeX\else
  % xetex/luatex font selection
\fi
% Use upquote if available, for straight quotes in verbatim environments
\IfFileExists{upquote.sty}{\usepackage{upquote}}{}
\IfFileExists{microtype.sty}{% use microtype if available
  \usepackage[]{microtype}
  \UseMicrotypeSet[protrusion]{basicmath} % disable protrusion for tt fonts
}{}
\makeatletter
\@ifundefined{KOMAClassName}{% if non-KOMA class
  \IfFileExists{parskip.sty}{%
    \usepackage{parskip}
  }{% else
    \setlength{\parindent}{0pt}
    \setlength{\parskip}{6pt plus 2pt minus 1pt}}
}{% if KOMA class
  \KOMAoptions{parskip=half}}
\makeatother
\usepackage{graphicx}
\makeatletter
\newsavebox\pandoc@box
\newcommand*\pandocbounded[1]{% scales image to fit in text height/width
  \sbox\pandoc@box{#1}%
  \Gscale@div\@tempa{\textheight}{\dimexpr\ht\pandoc@box+\dp\pandoc@box\relax}%
  \Gscale@div\@tempb{\linewidth}{\wd\pandoc@box}%
  \ifdim\@tempb\p@<\@tempa\p@\let\@tempa\@tempb\fi% select the smaller of both
  \ifdim\@tempa\p@<\p@\scalebox{\@tempa}{\usebox\pandoc@box}%
  \else\usebox{\pandoc@box}%
  \fi%
}
% Set default figure placement to htbp
\def\fps@figure{htbp}
\makeatother
% definitions for citeproc citations
\NewDocumentCommand\citeproctext{}{}
\NewDocumentCommand\citeproc{mm}{%
  \begingroup\def\citeproctext{#2}\cite{#1}\endgroup}
\makeatletter
 % allow citations to break across lines
 \let\@cite@ofmt\@firstofone
 % avoid brackets around text for \cite:
 \def\@biblabel#1{}
 \def\@cite#1#2{{#1\if@tempswa , #2\fi}}
\makeatother
\newlength{\cslhangindent}
\setlength{\cslhangindent}{1.5em}
\newlength{\csllabelwidth}
\setlength{\csllabelwidth}{3em}
\newenvironment{CSLReferences}[2] % #1 hanging-indent, #2 entry-spacing
 {\begin{list}{}{%
  \setlength{\itemindent}{0pt}
  \setlength{\leftmargin}{0pt}
  \setlength{\parsep}{0pt}
  % turn on hanging indent if param 1 is 1
  \ifodd #1
   \setlength{\leftmargin}{\cslhangindent}
   \setlength{\itemindent}{-1\cslhangindent}
  \fi
  % set entry spacing
  \setlength{\itemsep}{#2\baselineskip}}}
 {\end{list}}
\usepackage{calc}
\newcommand{\CSLBlock}[1]{\hfill\break\parbox[t]{\linewidth}{\strut\ignorespaces#1\strut}}
\newcommand{\CSLLeftMargin}[1]{\parbox[t]{\csllabelwidth}{\strut#1\strut}}
\newcommand{\CSLRightInline}[1]{\parbox[t]{\linewidth - \csllabelwidth}{\strut#1\strut}}
\newcommand{\CSLIndent}[1]{\hspace{\cslhangindent}#1}
\setlength{\emergencystretch}{3em} % prevent overfull lines
\providecommand{\tightlist}{%
  \setlength{\itemsep}{0pt}\setlength{\parskip}{0pt}}
\usepackage{booktabs}
\usepackage{longtable}
\usepackage{array}
\usepackage{multirow}
\usepackage{wrapfig}
\usepackage{float}
\usepackage{colortbl}
\usepackage{pdflscape}
\usepackage{tabu}
\usepackage{threeparttable}
\usepackage{threeparttablex}
\usepackage[normalem]{ulem}
\usepackage{makecell}
\usepackage{xcolor}
\usepackage{bookmark}
\IfFileExists{xurl.sty}{\usepackage{xurl}}{} % add URL line breaks if available
\urlstyle{same}
\hypersetup{
  pdftitle={Münsingen-Rain necropolis (Bern, Switzerland)},
  pdfauthor={Thomas Huet},
  hidelinks,
  pdfcreator={LaTeX via pandoc}}

\title{Münsingen-Rain necropolis (Bern, Switzerland)}
\usepackage{etoolbox}
\makeatletter
\providecommand{\subtitle}[1]{% add subtitle to \maketitle
  \apptocmd{\@title}{\par {\large #1 \par}}{}{}
}
\makeatother
\subtitle{A quantative study of Late Iron Age fibulae}
\author{Thomas Huet}
\date{janvier 2026}

\begin{document}
\maketitle

\begin{center}\rule{0.5\linewidth}{0.5pt}\end{center}

\section{Introduction}\label{introduction}

Munsingen-Rain is a Late Iron Age necropolis composed of \emph{circa}
220 graves, and 300 bronze and iron fibulae. The necropolis became the
favored focus for a wide range of experimental investigations, of a
typological, chronological, costume historical, art historical and
sociological nature (Müller, Jud, and Alt 2008). Here, we will focus on
the statistical analysis of fibulae, exploring the \{\texttt{archdata}\}
R package (Carlson and Roth 2021) (\ldots)

\begin{figure}[H]

{\centering \includegraphics[width=0.5\linewidth]{munsingen_fib_measures} 

}

\caption{Fibulae measurements (Hodson, 1970)}\label{fig:munsingenfib}
\end{figure}

Hodson (1970) took numerous measurements on the fibulae (Fig.
@ref(fig:munsingenfib)). This report aims to realise their statistical
analysis (\ldots)

\section{Dataset description}\label{dataset-description}

In the studied dataset, there are 30 fibulae described by 14
quantitative variables (FL, BH, BFA, FA, CD, BRA, etc.) (\ldots)

\begin{longtable}[t]{lrrrrr}
\caption{\label{tab:tabquantiles}Distribution by quantiles of fibulae measurments}\\
\toprule
 & 0\% & 25\% & 50\% & 75\% & 100\%\\
\midrule
FL & 9.0 & 19.25 & 21.50 & 28.750 & 94.0\\
BH & 7.0 & 15.00 & 15.50 & 18.000 & 26.0\\
BFA & 1.0 & 1.00 & 2.00 & 4.000 & 7.0\\
FA & 6.0 & 8.00 & 8.00 & 9.000 & 10.0\\
CD & 4.0 & 6.00 & 7.00 & 9.000 & 16.0\\
\addlinespace
BRA & 1.0 & 1.00 & 2.00 & 3.750 & 7.0\\
ED & 2.0 & 5.00 & 8.00 & 9.750 & 14.0\\
FEL & 0.0 & 4.00 & 7.00 & 11.000 & 50.0\\
C & 8.0 & 11.25 & 15.00 & 18.000 & 50.0\\
BW & 2.0 & 4.00 & 5.65 & 8.175 & 17.6\\
\addlinespace
BT & 1.4 & 3.05 & 3.85 & 4.775 & 7.7\\
FEW & 0.0 & 1.90 & 2.50 & 3.900 & 8.6\\
Coils & 3.0 & 4.00 & 6.00 & 6.000 & 22.0\\
Length & 26.0 & 41.75 & 49.50 & 59.750 & 128.0\\
\bottomrule
\end{longtable}

The Tab. @ref(tab:tabquantiles) resumes the distribution of fibulae
measurements by quantiles (\ldots)

\begin{figure}[H]

{\centering \includegraphics[width=0.7\linewidth]{2_Rmarkdown_Practice_files/figure-latex/histlength-1} 

}

\caption{Kernel Density Plot of Length}\label{fig:histlength}
\end{figure}

The histogram of the fibulae length (Fig. @ref(fig:histlength)) shows a
`L' shape (\ldots)

\section{Dataset exploration}\label{dataset-exploration}

\begin{figure}[H]

{\centering \includegraphics[width=0.7\linewidth]{2_Rmarkdown_Practice_files/figure-latex/correspanal-1} 

}

\caption{Correspondance Analysis (CA) of the dataset}\label{fig:correspanal}
\end{figure}

The Correspondence Analysis (Fig. @ref(fig:correspanal)) shows 61 \% of
the total variance.The point cloud shape is spherical except for one
individual (Mno: Hallstatt) and a variable (number of Coils) (\ldots)

\section{Statistical tests}\label{statistical-tests}

Following the Shapiro-Wilk normality test, the distribution of the
fibulae length is not normal as the Fig. @ref(fig:histlength) shown it.
It means (\ldots)

\section{Conclusion}\label{conclusion}

The statistical analysis of the dataset shows (\ldots)

\section*{References}\label{references}
\addcontentsline{toc}{section}{References}

\phantomsection\label{refs}
\begin{CSLReferences}{1}{0}
\bibitem[\citeproctext]{ref-Carlson21}
Carlson, David L., and Georg Roth. 2021. \emph{Archdata: Example
Datasets from Archaeological Research}.
\url{https://CRAN.R-project.org/package=archdata}.

\bibitem[\citeproctext]{ref-Hodson70}
Hodson, Frank Roy. 1970. {``Cluster Analysis and Archaeology: Some New
Developments and Applications.''} \emph{World Archaeology} 1 (3):
299--320.

\bibitem[\citeproctext]{ref-Muller08}
Müller, Felix, Peter Jud, and Kurt W. Alt. 2008. {``Artefacts, Skulls
and Written Sources: The Social Ranking of a Celtic Family Buried at
Münsingen-Rain.''} \emph{Antiquity} 82 (316): 462--69.
\url{https://doi.org/10.1017/S0003598X00096940}.

\end{CSLReferences}

\end{document}
